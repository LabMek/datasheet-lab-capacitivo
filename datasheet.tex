%% Datasheet ejemplo: Sensor Capacitivo de Nivel
\documentclass[10pt]{datasheet}

\usepackage[utf8]{inputenc}
\usepackage[spanish]{babel}
\usepackage{isodate}
\usepackage{tikz}
\usepackage{pgfplots}
\usepackage{circuitikz}
\usepackage{tabularx, threeparttable, multirow}
\usetikzlibrary{calc}

\title{Sensor Capacitivo de Nivel}
\author{Andres Morales Martínez}
\date{2025-ii}
\revision{Revisión 1}
\companylogo{
	\includegraphics[width=0.3\linewidth]{imgs/logo/stark.png}
}

\begin{document}
\maketitle

\section{Aplicaciones}
\begin{itemize}
    \item Medición de nivel de agua en tanques portátiles.
    \item Compatible con agua potable y agua destilada.
    \item Monitoreo en sistemas de bajo consumo con baterías de 9 V.
\end{itemize}

\section{Descripción General}
El sensor capacitivo de nivel detecta la variación en la constante dieléctrica 
del medio (aire, agua, aceite, etc.) para determinar el nivel de llenado en un depósito.  
Se caracteriza por su bajo consumo de energía y facilidad de integración en sistemas embebidos.  

\section{Principio de Funcionamiento}
El dispositivo consiste en un par de electrodos dispuestos de forma paralela que forman un capacitor.  
Cuando el medio entre los electrodos cambia (por ejemplo, de aire a agua), la capacitancia varía de acuerdo a:  

\[
C = \varepsilon_r \varepsilon_0 \frac{A}{d}
\]

donde $\varepsilon_r$ es la permitividad relativa del medio.  
Un circuito de acondicionamiento convierte esta variación de capacitancia en una señal eléctrica proporcional al nivel detectado.  

% Cambiamos a una sola columna desde aquí
\onecolumn

\section{Especificaciones Eléctricas}
\begin{table}[h]
\begin{threeparttable}
\caption{Características Eléctricas Básicas}
\begin{tabularx}{\textwidth}{l | c | c c c | c | X}
    \thickhline
    \textbf{Parámetro} & \textbf{Símbolo} & \textbf{Mín.} & \textbf{Típ.} & \textbf{Máx.} &
    \textbf{Unidad} & \textbf{Condiciones} \\
    \hline
    Tensión de alimentación & $V_{CC}$ & 7.5 & 9.0 & 12 & V & Fuente portátil (pila o batería) \\
    Señal de salida & $V_{out}$ & 0 & – & 5 & V & Señal analógica proporcional al nivel \\
    Impedancia de salida & $Z_{out}$ & – & 10 & – & k$\Omega$ & Medido a 1 kHz \\
    Consumo de corriente & $I_{CC}$ & – & 20 & 30 & mA & A 9 V \\
    \thickhline
\end{tabularx}
\end{threeparttable}
\end{table}

\section{Características Estáticas}

\subsection{Precisión}
La precisión se refiere a la cercanía entre el valor medido y el valor real del nivel.  
En este sensor depende de la estabilidad del circuito de conversión y de las propiedades dieléctricas del líquido.  

\subsection{Exactitud}
La exactitud se define como la capacidad del sensor de representar el nivel verdadero, considerando todos los errores sistemáticos.  
Puede ajustarse mediante calibración con líquidos de referencia (ej. agua destilada).  

\subsection{Linealidad}
La salida es aproximadamente lineal con el nivel del fluido dentro del rango de operación.  
La no linealidad típica es menor al 2\% del FSO.  

\subsection{Sensibilidad}
La sensibilidad indica el cambio en la señal de salida por cada milímetro de variación en el nivel.  
Depende del área de los electrodos y la geometría del sensor.  

\subsection{Rango de Entrada y Salida}
El rango de entrada corresponde al nivel de líquido entre 0 y 30 cm.  
El rango de salida es una señal analógica entre 0–5 V, escalada proporcionalmente.  

\subsection{FSO (Full Scale Output)}
El FSO corresponde al valor máximo de salida (5 V) cuando el tanque está en el nivel máximo especificado (30 cm).  

\section{Partes de Ensamble}
El sensor se compone de:  
\begin{itemize}
    \item Electrodos paralelos encapsulados en material dieléctrico.  
    \item Circuito de acondicionamiento de señal.  
    \item Salida analógica en conector de 3 pines (Vcc, GND, Vout).  
    \item Encapsulado resistente al agua (IP65).  
\end{itemize}

\section{Circuito de Conexión y Etapas}
\begin{figure}[h]
    \centering
    \begin{circuitikz}[scale=1]
        % Sensor capacitor
        \draw (0,0) to[C,l=$C_{sensor}$] (0,-2);
        % Excitation
        \draw (-2,0) node[left]{Excitación AC} to[sinusoidal voltage source] (0,0);
        % Amp
        \draw (0,-2) to[short] (2,-2) node[op amp, noinv input up, anchor=-](opamp) {};
        \draw (0,0) -| (opamp.+);
        \draw (2,-2) -- (opamp.-);
        % Output
        \draw (opamp.out) -- +(2,0) node[right]{$V_{out}$};
    \end{circuitikz}
    \caption{Etapas de conexión del sensor capacitivo de nivel}
\end{figure}

\section{Comentarios y Recomendaciones}
\begin{itemize}
    \item Para líquidos con baja constante dieléctrica, se recomienda calibrar antes de usar.  
    \item Evitar el uso en medios altamente conductivos sin aislamiento adecuado.  
    \item La linealidad puede mejorarse usando geometrías cilíndricas en lugar de placas planas.  
    \item Se sugiere emplear un filtro RC o un ADC con promedio digital para reducir el ruido.  
\end{itemize}

\end{document}
